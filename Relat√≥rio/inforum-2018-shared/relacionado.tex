
\section{Trabalho relacionado}
\label{relacionado}

A tecnologia da virtualização está em constante desenvolvimento, e o amadurecimento da tecnologia de conteinerização
foi acompanhado de muitos estudos comparativos entre a mesma e a tecnologia, mais madura, de máquinas virtuais.
No entanto, a maioria desses estudos senão todos, focam-se na análise comparativa do desempenho de ambas as tecnologias,
utilizando \textit{benchmarks} que avaliam, por exemplo, os custos da escrita em disco, 
da compressão de ficheiros, do desempenho de algoritmos no consumo da CPU, da gestão da memória e
seu dinamismo, do acesso à rede, etc.  procurando sempre realizar a comparação focando-se num recurso isolado e usando aplicações
centralizadas ou quase sempre centralizadas.

São exemplos destes estudos os seguintes: ????????

São também comuns comparações entre a execução de aplicações em modo nativo, directamente sobre o
hardware, com as mesmas aplicações a executarem em máquinas virtuais, como por exemplo os estudos referidos em  \cite{7095802}. ??????
........ No entanto este tipo de comparação não se adequa aos nossos objectivos, pois a comparação deveria
incidir sobre o comportamento de sistemas distribuídos escaláveis, o que exige o uso de múltiplos nós.

Existe uma tradição de utilização de {\conts} suportados directamente no sistema Linux para emulação de
redes. O exemplo mais conhecido é o Mininet~\cite{Lantz:2010:NLR:1868447.1868466}, cuja adequação e escalabilidade tem sido bastante
estudada~\cite{Handigol:2012:RNE:2413176.2413206}. No entanto, este sistema tem diferenças 
relativamente aos objectivos do nosso estudo. Por um lado, o sistema depende directamente de módulos
no Kernel do Linux que implementem \emph{switches} virtuais controlados directamente por OpenFlow.
Desta forma ficamos limitados a um único tipo de \emph{control plane}, isto é, um que se baseie em
SDN implementada com o protocolo OpenFlow. Por outro lado, tal como no nosso caso, os
estudos não abrangem a problemática do isolamento, isto é, não avaliam comparativamente
a utilização de {\conts} com ou sem VM de suporte. Adicionalmente, o ambiente de orquestração das
componentes da rede emulada é específico e provavelmente menos geral e flexível que o disponibilizado pela
plataforma Docker.






