
\section{Introdução}


Dada a complexidade e as dificuldades associadas ao teste de
protocolos distribuídos para controlo de equipamentos de rede num
contexto real, é comum recorrer a emulação para teste dos mesmos. Os
primeiros emuladores recorriam a máquinas reais, interligadas através
de um sistema de emulação do tempo de trânsito e da perda de pacotes
numa rede real \cite{ModelNet}.

Com o desenvolvimento das tecnologias de virtualização, seja esta através de
má\-qui\-nas virtuais ou de {\conts}, tornou-se atractivo substituir as
máquinas reais e os equipamentos de comutação por sistemas
virtualizados. Um dos sistemas mais populares que seguem a
abordagem com base em {\conts} é o sistema MiniNet~\cite{Lantz:2010:NLR:1868447.1868466}, para emulação de redes
estruturadas segundo o paradigma Software Defined Networking~\cite{sdnOverview}
e o protocolo OpenFlow~\cite{McKeown:2008:OEI:1355734.1355746}. Outros sistemas, que utilizam outros protocolos e outros
tipos de equipamentos de comutação, utilizam para o mesmo efeito
máquinas virtuais~\cite{1698543}.

No quadro de um projecto em que se pretende testar protocolos de
controlo de equipamentos de rede diferentes do
OpenFlow, e que utilizam protocolos distribuídos com semântica de falhas
bem definidas, da classe dos usados em geo-replicação e bases de dados distribuídas,
coloca-se o problema de decidir qual a tecnologia de virtualização a
utilizar: máquinas virtuais (VMs) ou {\conts}.

Este artigo apresenta um estudo empírico comparativo das tecnologias:
máquinas virtuais implementadas sobre o {\hiper}
ESXi da VMWare e {\conts} Docker suportados no interior de VMs.
A execução de {\conts} no interior de VMs, parecendo um contra-senso, 
não o é de facto pois a maioria das plataformas de grande escala (e.g., \clouds)
hoje acessíveis, não disponibiliza acesso nativo partilhado a
{\conts} por duas ordens de razões: segurança e isolamento dos diferentes \textit{tenants} quando usam {\conts}, 
e custo associado à existência de uma infraestrutura não virtualizada separada da "principal".

Para proceder ao estudo aqui apresentado foi necessário desenvolver de raiz um \textit{benchmark} novo,
pois os \textit{benchmarks} geralmente usados em estudos semelhantes, 
não estudam nem a escalabilidade do número de nós emulados,
nem a escalabilidade da solução quando existe um enorme volume de tráfego na rede. 
Tanto quanto é do nosso conhecimento, estes são traços que diferenciam este estudo de outros semelhantes.

O artigo está assim organizado. Na seção \ref{virtualizacao} apresenta-se uma breve comparação entre as tecnologias de virtualização de {\conts} e VMs. De seguida apresenta-se o \textit{benchmark} construído e clarifica-se a motivação para o seu desenvolvimento. Na quarta seção são expostos os testes realizados e apresentados os seus resultados. Posteriormente são discutidos os resultados obtidos na seção anterior e retiradas conclusões. Na seção \ref{relacionado} são analisados trabalhos relacionados. Por último apresentam-se as conclusões, 
refletindo sobre  possíveis melhoramentos e trabalho futuro.










