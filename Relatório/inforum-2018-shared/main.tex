\documentclass[a4paper]{llncs}
%

% Additional packages used
\usepackage[utf8]{inputenc}
\usepackage[portuges]{babel}
\usepackage{hyphenat}
\usepackage{amsmath}
\usepackage[pdftex]{graphicx}
\usepackage{tabularx}
\usepackage{comment}
\usepackage{calc}
\usepackage{pslatex}
\usepackage{listings}
\usepackage{url}
\usepackage{caption}
\usepackage{subcaption}
\usepackage{footnote}
\usepackage{float}
\captionsetup{compatibility=false}
\usepackage[section]{placeins}

% Packages needed for the table with alternating colors
\usepackage{xcolor}
\usepackage{colortbl}
\usepackage{booktabs}
\usepackage{tabu}

% Enumeration item customization
% \usepackage{enumitem}

% Inline lists
\usepackage{paralist}

%%%%%%%%%%%%%%%%%%%%%%%%%%%%%%%%%%%%

\newcommand{\alternativa}[1]{}
\newcommand{\ie}{\textit{i.e.},\ }
\newcommand{\eg}{\textit{e.g.},\ }
\newcommand{\etal}{\textit{et al.}\ }
\newcommand{\cf}{\textit{c.f.}\ }
\newcommand{\mapenc}{\textit{Map/Encap}}
\newcommand{\map}{sistema de \textit{mapping}}
\newcommand{\discutir}{\textbf{[por discutir] \ \ \ \ $\rightarrow$\ \ \ \ }}
\newcommand{\loc}{lo\-ca\-li\-za\-ção}
\newcommand{\run}{\textit{runtime }}


\newcommand{\lincont}{\textit{Linux container}}
\newcommand{\linconts}{\textit{Linux containers}}
\newcommand{\cont}{\textit{container}}
\newcommand{\conts}{\textit{containers}}
\newcommand{\Cont}{\textit{Container}}
\newcommand{\Conts}{\textit{Containers}}
\newcommand{\cloud}{\textit{cloud}}
\newcommand{\clouds}{\textit{clouds}}
\newcommand{\cldcomp}{\textit{Cloud Computing}}
\newcommand{\hiper}{\textit{hypervisor}}
\newcommand{\hipers}{\textit{hypervisors}}
\newcommand{\bench}{\textit{benchmark}}
\newcolumntype{P}[1]{>{\centering\arraybackslash}p{#1}}


%%%%%%%%%%%%%%%%%%%%%%%%%%%%%%%%%%%%

% Adjust distance between floats (figures and tables) and text
% \setlength{\belowcaptionskip}{-10pt}
% \setlength{\abovecaptionskip}{-10pt}

%%%%%%%%%%%%%%%%%%%%%%%%%%%%%%%%%%%%

\begin{document}

\title{{\Conts} versus Máquinas Virtuais para Emulação de Controlo de Redes}
\author{Bruno Anjos e Nuno Morais \\
\vspace{0.2cm}
Orientados por: Paulo Lopes e Jos\'{e} Legatheaux Martins}
%%%% list of authors for the TOC (use if author list has to be modified)
\tocauthor{}

\institute{NOVA LINCS, Departamento de Informática\\
           Faculdade de Ciências e Tecnologia, Universidade Nova de Lisboa\\
           2829--516 Caparica, Portugal\\
%
           \email{ \{b.anjos@campus., nm.morais@campus, paulo.lopes@, jose.legatheaux@\}fct.unl.pt} \\
           \url{http://di.fct.unl.pt}
           }
           


%\keywords{}

\maketitle              % typeset the title of the contribution


\begin{abstract}
Para o estudo do comportamento de sistemas sem recorrer à sua prévia implementação,
cada vez mais surge a necessidade de emulação dos mesmos, 
geralmente recorrendo a tecnologias de virtualização. 
Esta necessidade é particularmente aguda em sistemas distribuídos de larga escala.
Entre as soluções mais populares de virtualização estão os {\conts} e as máquinas virtuais (VMs).
Com a generalização do{ \cldcomp} e a cada vez maior 
% popularidade da 
utilização de virtualização, criou-se uma indústria à 
volta da otimização destas soluções e do \textit{deployment} 
de sistemas distribuídos escaláveis.
No caso das VMs a solução
% predominante 
mais popular em infraestruturas empresariais é a implementada pela VMWare com o {\hiper} ESXi,
e para {\conts} o Docker com o \textit{Docker Engine}. 
Neste projeto 
%foram testadas exaustivamente estas duas tecnologias 
%para emulação de um ambiente distribuído interligado. 
procurou-se analisar o desempenho da emulação de um sistema distribuído escalável
em ambas as tecnologias 
e concluir qual a melhor solução de virtualização para este tipo de sistema.

\end{abstract}

\label{introducao}

\section{Introdução}


Dada a complexidade e as dificuldades associadas ao teste de
protocolos distribuídos para controlo de equipamentos de rede num
contexto real, é comum recorrer a emulação para teste dos mesmos. Os
primeiros emuladores recorriam a máquinas reais, interligadas através
de um sistema de emulação do tempo de trânsito e da perda de pacotes
numa rede real \cite{ModelNet}.

Com o desenvolvimento das tecnologias de virtualização, seja esta através de
má\-qui\-nas virtuais ou de {\conts}, tornou-se atractivo substituir as
máquinas reais e os equipamentos de comutação por sistemas
virtualizados. Um dos sistemas mais populares que seguem a
abordagem com base em {\conts} é o sistema MiniNet~\cite{Lantz:2010:NLR:1868447.1868466}, para emulação de redes
estruturadas segundo o paradigma Software Defined Networking~\cite{sdnOverview}
e o protocolo OpenFlow~\cite{McKeown:2008:OEI:1355734.1355746}. Outros sistemas, que utilizam outros protocolos e outros
tipos de equipamentos de comutação, utilizam para o mesmo efeito
máquinas virtuais~\cite{1698543}.

No quadro de um projecto em que se pretende testar protocolos de
controlo de equipamentos de rede diferentes do
OpenFlow, e que utilizam protocolos distribuídos com semântica de falhas
bem definidas, da classe dos usados em geo-replicação e bases de dados distribuídas,
coloca-se o problema de decidir qual a tecnologia de virtualização a
utilizar: máquinas virtuais (VMs) ou {\conts}.

Este artigo apresenta um estudo empírico comparativo das tecnologias:
máquinas virtuais implementadas sobre o {\hiper}
ESXi da VMWare e {\conts} Docker suportados no interior de VMs.
A execução de {\conts} no interior de VMs, parecendo um contra-senso, 
não o é de facto pois a maioria das plataformas de grande escala (e.g., \clouds)
hoje acessíveis, não disponibiliza acesso nativo partilhado a
{\conts} por duas ordens de razões: segurança e isolamento dos diferentes \textit{tenants} quando usam {\conts}, 
e custo associado à existência de uma infraestrutura não virtualizada separada da "principal".

Para proceder ao estudo aqui apresentado foi necessário desenvolver de raiz um \textit{benchmark} novo,
pois os \textit{benchmarks} geralmente usados em estudos semelhantes, 
não estudam nem a escalabilidade do número de nós emulados,
nem a escalabilidade da solução quando existe um enorme volume de tráfego na rede. 
Tanto quanto é do nosso conhecimento, estes são traços que diferenciam este estudo de outros semelhantes.

O artigo está assim organizado. Na seção \ref{virtualizacao} apresenta-se uma breve comparação entre as tecnologias de virtualização de {\conts} e VMs. De seguida apresenta-se o \textit{benchmark} construído e clarifica-se a motivação para o seu desenvolvimento. Na quarta seção são expostos os testes realizados e apresentados os seus resultados. Posteriormente são discutidos os resultados obtidos na seção anterior e retiradas conclusões. Na seção \ref{relacionado} são analisados trabalhos relacionados. Por último apresentam-se as conclusões, 
refletindo sobre  possíveis melhoramentos e trabalho futuro.












\label{virtualizacao}

\section{Máquinas virtuais versus {\conts}}

A virtualização, nomeadamente sob a forma de Máquinas Virtuais (VM), é uma tecnologia que se encontra presente 
em múltiplos sistemas em produção, particularmente em centros de dados. Esta tem progredido maioritariamente devido à sua vasta utilização no contexto empresarial. As primeiras ocorrências de virtualização surgiram na década de 60 e atualmente ainda se encontram em desenvolvimento \cite{7095802}.

No entanto, recentemente surgiram novas tecnologias de virtualização com a intro\-du\-ção dos  {\linconts} em 2008 \cite{wiki-lxc}. 
Esta nova tecnologia trouxe consigo uma solução mais leve em termos dos recursos exigidos, sacrificando isolamento e segurança.
 
Uma das soluções mais populares de virtualização por {\conts} é disponibilizada pela plataforma Docker \cite{docker} e executada num ambiente Linux. A figura \ref{fig:vms-conts}, retirada de \cite{image-vms-conts}, apresenta uma visualização sintética das duas tecnologias e das principais diferenças, que são em seguida apresentadas em mais detalhe.

\begin{figure}[h!]
	\centering
	\includegraphics[width=\linewidth]{figures/docker-vs-virtual-machines.png}
	\caption{Arquitetura da virtualização em VMs e {\conts}} \cite{image-vms-conts}
	\label{fig:vms-conts}
\end{figure}


\subsection{Máquinas Virtuais}
Num ambiente de virtualização com recurso a máquinas virtuais o {\hiper} é o componente principal. Como podemos observar na figura \ref{fig:vms-conts}, este componente é fundamental pois é responsável por ocultar o \textit{hardware} real e disponibilizar sob a forma de \textit{hardware} virtual (que nalguns casos, mas não em todos, tem exatamente as mesmas propriedades que o real) os recursos afetados à máquina virtual. O isolamento entre máquinas virtuais é um ponto forte desta solução. Este isolamento garante (à partida) que se um serviço for comprometido, outras máquinas virtuais instanciadas no mesmo servidor, e até o próprio {\hiper} não serão também comprometidas. 

De uma forma simplificada, podemos dizer que a execução de aplicações numa VM decorre na sua maior parte sem intervenção do {\hiper}; 
contudo, a execução de certas instruções privilegiadas obriga à sua intervenção, o que contribui com algum \textit{overhead} na execução.

\subsection{{\Conts}} \label{section_conts}
A tecnologia dos {\conts} baseia-se numa virtualização ao nível do sistema de operação (SO), inicialmente baseada nos \textit{control groups} e \textit{kernel namespaces} presentes no Linux. Isto permite que várias instâncias de {\conts} possam ser executadas sobre o mesmo \textit{kernel}, evitando assim a camada de virtualização do \textit{hardware} presente nas máquinas virtuais (observável na Figura \ref{fig:vms-conts}). Assim, e ao contrário do que acontece nas VMs, a execução de tais instruções privilegiadas, {\eg} as utilizadas em leituras do disco, são efetuadas diretamente no SO do \textit{host}. A tecnologia Docker (\textit{Docker engine}) domina atualmente o mercado no que toca a este tipo de virtualização. No entanto a segurança da mesma ainda é um fator de preocupação devido à falta de isolamento entre processos. 

\discutir{rever isto, elaborar com a tabela, falar de Copy On write}

\begin{table}[h!]
\centering
    	\renewcommand{\arraystretch}{1.5}
	\setlength{\arrayrulewidth}{0.5mm}
	\setlength{\tabcolsep}{12pt}	
    \begin{tabular}{ | P{20mm} | P{40mm} | P{40mm} | } 
    \hline
    \textbf{Tópico} & \textbf{\Conts} & \textbf{Máquinas Virtuais} \\ \hline
    \hline
    \textbf{Virtualização} & Virtualização ao nível do sistema de operação & Virtualização ao nível do \textit{hardware} \\
    \hline
    \textbf{Inicialização} & Cerca de 500ms & Cerca de 20s \\
    \hline
    \textbf{Versatilidade} & Numa solução nativa é restrito ao ambiente Linux \footnotemark & Independente do sistema de operação \\
    \hline
    \textbf{Isolamento} & Baseado em \textit{namespaces} e \textit{Cgroups} & Isolamento total \\
    \hline
    \textbf{Maturidade da tecnologia} & Pouco empregue em contexto empresarial & Presente na maioria dos sistemas de virtualização em produção \\
    \hline
    \textbf{Chamadas de sistema} & Chamadas de sistema diretas ao SO do \textit{host} & Chamadas ao SO \textit{guest} e posteriormente ao {\hiper} \\
    \hline
    \textbf{Utilização de disco} & Apenas possui dependências dos executáveis e de um \textit{filesystem} & Possui todos os ficheiros presentes num sistema de operação \\
    \hline
\end{tabular}
\caption{Principais diferenças entre {\conts} e máquinas virtuais}
\end{table}


\footnotetext{O ambiente Docker está transportado para outros sistemas de operação mas geralmente recorre igualmente a máquinas virtuais, podendo ser considerada uma solução mista}

\subsection{Conclusões}

Em suma, as principais diferenças entre as duas opções de virtualização resumem-se no consumo de recursos por instância, isolamento entre instâncias e no nível de abstração. É importante salientar que, embora não sejam aspetos decisivos, o tempo de arranque, o espaço consumido em disco e ferramentas disponibilizadas pelas tecnologias {\eg} Docker Hub e VMWare vSphere são fatores a considerar ao escolher uma destas tecnologias.







\label{benchmarks}



\section{Benchmarks e metodologia de testes} \label{benchs}

O controlo de redes ao nível do chamado \emph{Control Plane} baseia-se, por razões relacionadas com a necessidade de um elevado desempenho e de simplicidade, em protocolos de coordenação entre as diferentes componentes de rede (e.g. \emph{switches, routers, controladores \dots}) com níveis de consistência do tipo ``consistência eventual''. Por essa razão, o estado propagado ou computado pelas diferentes componentes da rede transita entre estados consistentes através de períodos de inconsistência \cite{  } mais ou menos prolongados.






Para tal foi criado um \textit{benchmark} baseado na análise do tempo de execução do ciclo principal da aplicação. Fazendo variar o débito e o número de elementos do sistema 
distribuído é possível analisar a capacidade de ambas as tecnologias de emular de uma rede virtual e também observar quais os limites desta.

Foi desenvolvida uma aplicação em python cujo objetivo consiste na inserção periódica de tuplos numa base de dados mysql local e periodicamente enviar os mesmos agroupadamente para serem inseridos em bases de dados remotas, replicando assim a base de dados. 

Este formato de testes foi construído de raíz, pois o que foi verificado no trabalho relacionado (ver \ref{relac}), é que por norma os testes de \textit{performance} entre máquinas virtuais e containers são a um nível local e não implicam qualquer tipo de comunicação. Visando este projecto a comparação de \textit{performances} de diferentes ambientes a um nível distribuído, foi então desenhada uma aplicação que representasse um sistema desse mesmo tipo.

O sistema onde foram efetuados os \textit{benchmarks} em questão, tinha dois processadores \textit{Intel Xeon E5-2670 v3} a 2.30GHz com 12 núcleos físicos (24 lógicos) cada, 128GB de RAM DDR4 2400MHz e corria o Hipervisor VMWare ESXi 6.5 diretamente sobre o hardware.

Para os testes do ambiente de containers Docker, foi criada uma máquina virtual que permitisse os testes correrem num aspeto semelhante a um sistema cloud, onde os sistemas de containers correm sobre uma máquina virtual por questões de segurança. Para máquina em questão, dos recursos previamente mencionados foram disponibilizados os 12 núcleos físicos de ambos os processadores e 96GB de RAM. Não foi selecionada toda a RAM disponível pois escolheu-se deixar alguma disponível para o hipervisor. O sistema operativo de eleição para a máquina virtual dos testes foi o CentOS 7.

\subsection{Funcionamento da aplicação} \label{func}

\begin{itemize}

\item Seja D o débito da aplicação, a razão entre a quantidade de tuplos inseridos e uma unidade temporal. D = $\frac{msg}{minuto}$
\item Seja AG o fator de agregação das mensagens enviadas para as bases de dados remotas
\item Seja N o número de instâncias que compõe o sistema distribuído

\end{itemize}


O fluxo da aplicação consiste em três estados principais:

\begin{enumerate}
\item Sincronização
\begin{description}
\item A fase de sincronização possui como objetivo a coordenação das múltiplas instâncias da aplicação. Esta consiste no estabelecimento de ligações entre os diversos componentes do sistema. Após estabelecidas as ligações, é inserida uma entrada numa tabela de sincronização pertencente á base de dados elemento cujo endereço IP é o menor (\textit{master node}). Posteriormente são efetuadas listagens dessa tabela periodicamente, no momento em que o número de entradas for igual ao número de elementos do sistema distribuído é concluído que todos os elementos do sistema estão prontos para prosseguir para a seguinte fase de execução.
\end{description}

\item Ciclo principal
\begin{description}
\item No ciclo principal periodicamente (de acordo com D), será gerado um número aleatório de 0-100 e gerada uma sequência aleatória de carateres. Um tuplo composto pelos elementos referidos anteriormente será inserido na tabela da base de dados local. Após AG inserções locais serem executadas, são inserido os mesmos tuplos agregados nas restantes N - 1 bases de dados. Cada instância da aplicação irá executar estas inserções, resultando na tabela referida anteriormente replicada em N bases de dados.
\end{description}

\item Dessincronização
\begin{description}
\item Na fase de dessincronização todos os elementos do sistema distribuído irão remover o tuplo inserido durante a fase de sincronização (presente na tabela do \textit{master node}), em seguida irá efetuar a listagem dessa tabela até que esta esteja vazia. Nesse momento é concluída a fase de dessincronização.
\end{description}

\end{enumerate}







\label{testes}

\section{Testes}

Foram realizados diversos testes que estudam o impacto da variação do débito e do número de nós do sistema no tempo de
execução. Um conjunto de testes foi efetuado num modo que designamos por \textit{baseline}. 
Este consiste em eliminar toda a comunicação entre nós do sistema, visando analisar o \textit{overhead}
introduzido pela troca de mensagens (e consequentemente, da emulação da rede virtual). No entanto, a
actividade de geração e inserção de tuplos na base de dados é preservada, realizando-se um
número total equivalente de gerações e inserções de tuplos.


Em todos os testes foi fixado o fator de agregação ($AG = 100$), referido em \ref{benchs}, uma vez que se concluiu
que este parâmetro apenas é relevante para valores baixos (de 1 a 5).

Todos os testes foram executados em ambos os ambientes ({\conts} e VMs). Os parâmetros de teste são os seguintes:
\begin{itemize}

\item Variação de débito com 15 nós e todas as inserções locais (\textbf{teste 1}: {\conts}, $D$ variável, $N = 15$ e modo \textit{baseline})
\item Variação do número de nós com cada nó a gerar 5000 tuplos por minuto que insere $N$ vezes localmente (\textbf{teste 2}: {\conts}, $D$ = 5000, $N$ variável, e modo \textit{baseline})

\item Variação de débito (\textbf{teste 3}: {\conts}, $D$ variável, $N = 15$)
\item Variação do número (\textbf{teste 4}: {\conts}, $D$ = 5000, $N$ variável)

\end{itemize}

O comportamento esperado do sistema varia de acordo com o número de operações que este executa por unidade de tempo. Este número é,
no essencial, proporcional ao número de inserções na base de dados.
A equação que define o número total de inserções por unidade de tempo é a seguinte:

\begin{equation}
\label{eq:operacoes}
 I = N^2 * D
\end{equation}

% Sendo que N corresponde ao número de nós do \textit{benchmark} e D ao débito de geração de tuplos por nó.

\subsection{Resultados}

Nesta secção serão apresentados os resultados obtidos pelos testes apresentados acima. 
Cada gráfico relaciona o parâmetro que varia com o desvio (em percentagem) do tempo de execução esperado,
\ie o tempo definido pelo débito $D$ admitindo que as operações são realizadas sem atrasos.

	
\begin{figure}[h!]
	\centering
	\includegraphics[width=0.9\linewidth]{figures/graphs/baseline_insertions}
	\caption{\textbf{Teste 1} ({\conts} e VMs, $D$ variável, $N = 15$ e modo \textit{baseline})}
	\label{fig:results-ins-baseline}
\end{figure}

\begin{figure}[h!]
	\centering
	\includegraphics[width=0.9\linewidth]{figures/graphs/baseline_numNodes}
	\caption{\textbf{Teste 2} ({\conts} e VMs, $D = 5000$, $N$ variável, e modo \textit{baseline})}
	\label{fig:results-nos-baseline}
\end{figure}

\begin{figure}[h!]
	\centering
	\includegraphics[width=0.9\linewidth]{figures/graphs/insertions}
	\caption{\textbf{Teste 3} ({\conts} e VMs, $D$ variável, $N = 15$)}
	\label{fig:results-insertions}
\end{figure}

\begin{figure}[h!]
	\centering
	\includegraphics[width=0.9\linewidth]{figures/graphs/numNodes}
	\caption{\textbf{Teste 4} ({\conts} e VMs, $D = 5000$, $N$ variável)}
	\label{fig:results-nos}
\end{figure}

%Nas figuras \ref{fig:docker-ins} e \ref{fig:vms-ins} encontra-se representado o comportamento de ambos os ambientes de emulação fazendo variar as inserções por minuto. Pode-se então verificar que fixando o número de nós, os testes exibem um incremento linear, tal como esperado de acordo com a equação anteriormente apresentada. A diferença entre as duas soluções é insignificante.


%As figuras \ref{fig:docker-nos} e \ref{fig:vms-nos} apresentam os resultados dos testes em que se estuda o impacto da variação do número de instâncias. Verifica-se que no caso dos {\conts} o \textit{runtime} varia de acordo com o aumento de operações.



\label{discussao}

\section{Discussão de resultados}
\subsection{\emph{Baseline}}

Na Figura \ref{fig:results-ins-baseline} encontra-se representado o comportamento de ambos os ambientes de emulação fazendo variar as gerações e inserções por minuto sem comunicação. Pode-se verificar que fixando o número de nós, os testes exibem um incremento linear, tal como esperado de acordo com a equação anteriormente apresentada. A diferença entre as duas soluções manifesta-se quando o débito é superior a 180.000 tuplos gerados por minuto. Esta deve-se, provavelmente, à camada adicional de virtualização ({\conts} sobre uma VM) que introduz \textit{overhead} suplementar.

O gráfico da Figura \ref{fig:results-nos-baseline} não apresentam qualquer desvio do tempo de execução esperado. Visto isto, nos gráficos onde é introduzida a comunicação, caso os resultados se distanciem dos obtidos nos testes \textit{baseline}, conclui-se que a diferença se deve á emulação da rede virtual.

\subsection{Variação do débito $D$}

Observando a Figura \ref{fig:results-insertions} verifica-se que mantendo um número constante de nós o desvio exibe um incremento linear, de acordo com a equação (\ref{eq:operacoes}). Comparando os dois gráficos, observa-se que a utilização das duas tecnologias de virtualização testadas não modifica significativamente os resultados obtidos. Desse modo pode concluir-se que o fator que limita a fiabilidade da emulação tem o mesmo peso nos testes realizados nos dois ambientes de virtualização.

\subsection{Variação do número de nós $N$}

Na figura \ref{fig:results-nos} verifica-se que a partir de 32 nós o desvio assemelha-se a uma curva quadrática, semelhante á equação (\ref{eq:operacoes}). Em contraste, a Figura \ref{fig:vms-nos} não apresenta o mesmo comportamento, mantendo-se o desvio 
quase nulo. Uma possível explicação deste resultado, é o \textit{overhead} suplementar introduzido devido aos {\conts} estarem a executar sobre uma máquina virtual. Essa arquitetura parece prejudicar a solução de {\conts} usada nos ambientes de {\cloud} para assegurar um nível superior de isolamento. Outra possível origem da disparidade dos dois resultados é uma virtualização de rede de qualidade inferior no caso do \textit{Docker Engine}.

Conclui-se, com base nestes gráficos, que a solução de máquinas virtuais possui, neste ambiente, uma escalabilidade superior quando o número de nós aumenta.Tal constitui, aparentemente, um resultado surpreendente e contrário à intuição geralmente prevalente na literatura.

\label{relacionado}

\section{Trabalho relacionado}
\label{relacionado}

A tecnologia da virtualização está em constante desenvolvimento, e o amadurecimento da tecnologia de conteinerização
foi acompanhado de muitos estudos comparativos entre a mesma e a tecnologia, mais madura, de máquinas virtuais.
No entanto, a maioria desses estudos senão todos, focam-se na análise comparativa do desempenho de ambas as tecnologias,
utilizando \textit{benchmarks} que avaliam, por exemplo, os custos da escrita em disco, 
da compressão de ficheiros, do desempenho de algoritmos no consumo da CPU, da gestão da memória e
seu dinamismo, do acesso à rede, etc.  procurando sempre realizar a comparação focando-se num recurso isolado e usando aplicações
centralizadas ou quase sempre centralizadas.

São exemplos destes estudos os seguintes: ????????

São também comuns comparações entre a execução de aplicações em modo nativo, directamente sobre o
hardware, com as mesmas aplicações a executarem em máquinas virtuais, como por exemplo os estudos referidos em  \cite{7095802}. ??????
........ No entanto este tipo de comparação não se adequa aos nossos objectivos, pois a comparação deveria
incidir sobre o comportamento de sistemas distribuídos escaláveis, o que exige o uso de múltiplos nós.

Existe uma tradição de utilização de {\conts} suportados directamente no sistema Linux para emulação de
redes. O exemplo mais conhecido é o Mininet~\cite{Lantz:2010:NLR:1868447.1868466}, cuja adequação e escalabilidade tem sido bastante
estudada~\cite{Handigol:2012:RNE:2413176.2413206}. No entanto, este sistema tem diferenças 
relativamente aos objectivos do nosso estudo. Por um lado, o sistema depende directamente de módulos
no Kernel do Linux que implementem \emph{switches} virtuais controlados directamente por OpenFlow.
Desta forma ficamos limitados a um único tipo de \emph{control plane}, isto é, um que se baseie em
SDN implementada com o protocolo OpenFlow. Por outro lado, tal como no nosso caso, os
estudos não abrangem a problemática do isolamento, isto é, não avaliam comparativamente
a utilização de {\conts} com ou sem VM de suporte. Adicionalmente, o ambiente de orquestração das
componentes da rede emulada é específico e provavelmente menos geral e flexível que o disponibilizado pela
plataforma Docker.








\label{fim}

\section{Conclusões e trabalho futuro}



%Surgiram alguns percalços nomeadamente na utilização do motor de base de dados, pois sabendo que os \textit{benchmarks} são testes de performance, há uma necessidade de otimizar o motor de base de dados de forma a introduzir o menor \textit{overhead} possível, pois o objectivo em questão é verificar a performance da emulação do sistema consoante as variáveis mencionadas em \ref{benchs} e evitar um \textit{bottleneck} comum entre os testes. 

%Foi discutido o aumento do número de \textit{threads} de escrita na base de dados visando melhorias ao nível do paralelismo, mas a ideia acabou por ser descartada, devido ao número de núcleos lógicos atribuído a cada máquina virtual ser único a partir de 24 instâncias. O maior imprevisto que pode ser tido em conta num trabalho futuro, surgiu aquando dos testes nas máquinas virtuais, em que, na fase de sincronização (ver \ref{func}), cada máquina virtual demorava bastante tempo (por volta de 5 minutos) a establecer as conexões para as restantes instâncias. Após algum aprofundamento da questão, chegou-se à conclusão de que estava relacionado com a resolução de nomes das restantes máquinas. Foram implementadas algumas possíveis soluções, tais como, desativar a resolução de nomes do motor da base de dados e adicionar os endereços e os nomes ao ficheiro de resolução de nomes utilizado pelo sistema operativo, mas em nenhuma das soluções se verificou uma melhoria na situação em questão. De qualquer das formas esta questão não foi mais aprofundada pois a fase que dita a performance de um determinado teste (ciclo principal, ver \ref{func}) não é influenciada.

%Surgiram também percalços relativamente á fase de sincronização do sistema, visto que a implementação inicial permitia que instâncias começassem a executar o seu ciclo principal sem que as outras estivessem no final da fase de sincronização. Visto isto, foi implementada uma solução que emula um \textit{filesystem} partilhado entre as várias instâncias (a tabela de sincronização do \textit{master node}).



Neste projeto foram avaliadas duas soluções de emulação de sistemas distribuídos escaláveis. 
Em ambos os casos foi testada a eficiência da implementação da rede virtual através de testes 
que simulavam uma troca intensiva de mensagens pela rede. 

No ambiente de testes usado, verificou-se um comportamento semelhante das duas soluções de
virtualização quando não se utiliza a emulação da rede. Em contrapartida, quando esta emulação
é usada, as VMs têm um desempenho superior quando o número de nós aumenta. A explicação deste resultado
necessita de ser aprofundada, mas tal não nos impede de retirar desde já várias conclusões imediatas.

Para começar, é importante frisar que o desenvolvimento do projeto provou ser mais ágil na plataforma Docker. 
Em contraste, no ambiente de VMs foi necessário criar vários \textit{scripts} auxiliares para
automatizar o processo de instalação e de realização dos testes.

No decorrer do desenvolvimento foram estudadas e testadas várias ferramentas disponibilizadas pelas plataformas, 
como por exemplo Docker Swarm, Docker Compose, bem como o sistema de operação 
Photon OS da VMWare. 
No entanto, estas ferramentas não foram utilizadas pois foi possível alcançar um controlo mais fino através de soluções
específicas desenvolvidas por nós.

Surgiram obstáculos relativos á utilização do MySQL devido á natureza  ACID
das transações, o que provocou um \textit{bottleneck} na utilização do disco.
Foi discutido o aumento do número de \textit{threads} de escrita na base de dados visando melhorias do desempenho. No entanto,
a ideia foi descartada 
devido ao facto de que o número de \textit{threads} total ser já consideravelmente maior que o número de núcleos do processador. 
Visto que a persistência da base de dados não era um fator importante, 
foi utilizado um \textit{ramdisk} com o objetivo de remover este \textit{bottleneck}.

Um outro imprevisto surgiu na fase de sincronização das máquinas virtuais que
demorava cerca de 40 vezes mais do que no 
Docker, devido a problemas na resolução de nomes.
Este imprevisto resultou no atraso da obtenção de resultados na fase mais crítica do trabalho.

Existe lugar para trabalho futuro imediato, nomeadamente testar os {\conts} sobre PhotonOS
e sem VM intermédia. Estes testes permitirão aprofundar a origem do \textit{overhead} suplementar verificado,
assim como aclarar a problemática da viabilidade da plataforma Docker com ou sem isolamento forte.
Relativamente às ferramentas desenvolvidas, a documentação das mesmas deve ser melhorada antes de as
disponibilizarmos publicamente. Adicionalmente, a bateria de testes poderá também ser alargada de modo a
clarificar o comportamento do sistema quando o desvio do tempo de execução começa a aumentar.

Tal como alertados pelos orientadores, a investigação em sistemas reveste-se de inúmeros obstáculos
e problemas concretos a resolver, antes de se chegarem a conclusões credíveis e verificáveis publicamente
sobre o comportamento dos sistemas, o que foi para nós algo
completamente novo.


%\tiny

\bibliographystyle{abbrv}

\bibliography{bibliography}


\end{document}