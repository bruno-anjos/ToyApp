
\section{Introdução}

\subsection{XXXXXXXXXX }

{ \color{blue} Explicar a razão de ser do trabalho, porque foi necessário, como se fez e dar uma
breve panoramica dos resultados - usar parte do abstract que passará a ser mais resumido.}



Dada a complexidade e as dificuldades associadas ao teste de
protocolos distribuídos para controlo de equipamentos de rede num
contexto real, é comum recorrer a emulação para teste dos mesmos. Os
primeiros emuladores recorriam a máquinas reais, interligadas através
de um sistema de emulação do tempo de trânsito e da perda de pacotes
numa rede real [ModelNet].

Com o desenvolvimento das tecnologias de virtualização, seja esta através de
máquinas virtuais ou de \conts, tornou-se atractivo substituir as
máquinas reais e os equipamentos de comutação por sistemas
virtualizados. Um dos sistemas mais populares que seguem esta
abordagem é o sistema MiniNet \cite{Lantz:2010:NLR:1868447.1868466}, para emulação de redes
controladas com a abordagem Software Defined Networking e o protocolo
OpenFlow. Outros sistemas, que utilizam outros protocolos e outros
tipos de equipamentos de comutação, utilizam para o mesmo efeito
máquinas virtuais [ Cisco ].

No quadro de um projecto em que se pretende testar protocolos de
controlo de equipamentos de rede mais complexos que os switches
OpenFlow, e utilizando protocolos distribuídos com semântica de falhas
bem definidas, da classe dos usados em bases de dados distribuídas,
coloca-se o problema de decidir qual a tecnologia de virtualização a
utilizar: máquinas virtuais (VMs) ou {\conts} implementados pelo sistema
Docker.

Este artigo apresenta um estudo empírico comparativo das tecnologias:
máquinas virtuais implementadas sobre o hipervisor 
ESXi da VMWare e containers Docker suportados quer no interior de VMs, quer em modo nativo.
A execução de containers no interior de VMs, parecendo um contrasenso, 
não o é de facto pois a maioria das plataformas de grande escala (e.g., \clouds)
hoje acessíveis não disponibiliza acesso nativo aos utilizadores.

Foi então realizado um plano de trabalhos, visando definir os objectivos do projecto de investigação e estabelecer alguns \textit{deadlines} e a respectiva alocação de tempo para cada tarefa.
\vspace{5mm}

\textbf{Semana 1 (8/4/2018 : 15/4/2018)}

\begin{enumerate}
\item Automatizar o deployment da aplicação de teste para  executar múltiplos testes num quadro simples de um container por nó
\end{enumerate}

\textbf{Semanas 2 a 4 (15/4/2018 : 6/5/2018)}


\begin{enumerate}
\item Estudar as tecnologias presentes e descobrir as diferenças fundamentais entre as mesmas
\item Formular uma hipótese sobre qual a  raiz das diferenças de desempenho
\item Identificar os parâmetros e configurações a medir que permitam efetuar a comparação de resultados de forma rigorosa

\end{enumerate}

\textbf{Semana 5 a 7 (6/5/2018 : 27/5/2018)}

\begin{enumerate}
\item Definir cenários (um ou mais servidores) e parâmetros de teste
\item Executar os testes
\item Analisar os resultados obtidos
\end{enumerate}


\textbf{Semana 8 a 9 (27/5/2018 : 10/6/2018)}

\begin{enumerate}
\item Consolidar a análise dos dados obtidos
\item Elaboração do relatório
\end{enumerate}
